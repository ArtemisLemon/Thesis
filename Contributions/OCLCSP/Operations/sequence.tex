\newpage\subsection{src.prepend(e)}

Consider the expression \inlineocl{src.prepend(e)}, where \inlineocl{src} is an expression resulting in a collection of objects or integers and \inlineocl{e} is an expression of resulting in an object or int.
% Formal Definition:
% \begin{equation}\label{def:prepend}
%     prepend(X,Y,exp) \iff
% \end{equation}

Reformulation using Global Constraints:
\begin{equation}\label{csp:prepend}
    prepend(X,Y,z):
    \begin{cases}
        Y = z\;\mathbin{\|}\;X
    \end{cases}
\end{equation}


\newpage\subsection{src.append(e)}

Consider the expression \inlineocl{src.append(e)}, where \inlineocl{src} is an expression resulting in a collection of objects or integers and \inlineocl{e} is an expression of resulting in an object or int.
% Formal Definition:
% \begin{equation}\label{def:append}
%     append(X,Y,exp) \iff
% \end{equation}

Reformulation using Global Constraints:
\begin{equation}\label{csp:append}
    append(X,Y,z):
    \begin{cases}
        Y' = X\;\mathbin{\|}\;z\\
        asSequence(Y',Y)
    \end{cases}
\end{equation}
% \begin{equation}\label{csp:append_seq}
%     append(X,Y,z):
%     \begin{cases}
%         Y = X\;\mathbin{\|}\;z
%     \end{cases}
% \end{equation}




\newpage\subsection{src.insertAt(i,e)}

Consider the expression \inlineocl{src.insertAt(i,e)}, where \inlineocl{src} is an expression resulting in a collection of objects or integers, \inlineocl{i} is an expression resulting in an integer, and \inlineocl{e} is an expression of resulting in an object or int.
% Formal Definition:
% \begin{equation}\label{def:insertAt}
%     insertAt(X,Y,exp) \iff
% \end{equation}

Reformulation using Global Constraints:
\begin{equation}\label{csp:insertAt}
    insertAt(X,Y,z,p):
    \begin{cases}
        p'_i = i + \llbracket i \geq z \rrbracket\\
        element(x_i,Y,p'_i)\\
        element(z,Y,p)
    \end{cases}
\end{equation}

\begin{figure}[!ht]
    \centering
    \includegraphics[trim={0 30 0 0},width=1\linewidth]{Contributions/OCLCSP/Operations/figures/insertAt.png}
    \caption{example solution for insert at}
    \label{fig:subsequence}
\end{figure}

% \newpage\subsection{Sub-Sequence}
\newpage\subsection{src.subSequence(s,f)}

Consider the expression \inlineocl{src.subSequence(e)}, where \inlineocl{src} is an expression resulting in a collection of objects or integers and \inlineocl{s,f} are expressions of resulting integers giving the start and finish of the sub-sequence in the sequence.
% Formal Definition:
% \begin{equation}\label{def:subSequence}
%     subSequence(X,Y,exp) \iff
% \end{equation}

Reformulation using Global Constraints:
\begin{equation}\label{csp:subSequence}
    subSequence(X,Y,s,f):
    \begin{cases}
        X' = d\;\mathbin{\|}\;X\\
        p_1 = s+1\\
        p_i = (s+i)*\llbracket s+i\leq j\rrbracket\\
        element(y_i,X',p_i)
    \end{cases}
\end{equation}

\begin{figure}[!ht]
    \centering
    \includegraphics[trim={0 30 0 0},width=1\linewidth]{Contributions/OCLCSP/Operations/figures/subSeq.png}
    \caption{example solution for sub-sequence}
    \label{fig:subsequence}
\end{figure}

% \newpage\subsection{At}
\newpage\subsection{src.at(p)}

Consider the expression \inlineocl{src.at(p)}, where \inlineocl{src} is an expression resulting in a collection of objects or integers and \inlineocl{p} is an expression of resulting in an integer.
% Formal Definition:
% \begin{equation}\label{def:at}
%     at(X,Y,exp) \iff
% \end{equation}

Reformulation using Global Constraints:
\begin{equation}\label{csp:at}
    at(X,y,p):
    \begin{cases}
        element(y,X,p)
    \end{cases}
\end{equation}

% \newpage\subsection{Index Of}
\newpage\subsection{src.indexOf(p)}

Consider the expression \inlineocl{src.indexOf(p)}, where \inlineocl{src} is an expression resulting in a collection of objects or integers and \inlineocl{p} is an expression of resulting in an integer.
% Formal Definition:
% \begin{equation}\label{def:indexOf}
%     indexOf(X,Y,exp) \iff
% \end{equation}

Reformulation using Global Constraints:
\begin{equation}\label{csp:indexOf}
    indexOf(X,y,p):
    \begin{cases}
        element(y,X,p)
    \end{cases}
\end{equation}

% \newpage\subsection{First}
\newpage\subsection{src.first()}

Consider the expression \inlineocl{src.first()}, where \inlineocl{src} is an expression resulting in a collection of objects or integers.
% Formal Definition:
% \begin{equation}\label{def:first}
%     first(X,Y,exp) \iff
% \end{equation}

Reformulation using Global Constraints:
\begin{equation}\label{csp:first}
    first(X,y):
    \begin{cases}
        y = x_1
    \end{cases}
\end{equation}

% \newpage\subsection{Last}
\newpage\subsection{src.last()}

Consider the expression \inlineocl{src.last()}, where \inlineocl{src} is an expression resulting in a collection of objects or integers.
% Formal Definition:
% \begin{equation}\label{def:last}
%     last(X,Y,exp) \iff
% \end{equation}

Reformulation using Global Constraints:
\begin{equation}\label{csp:last}
    last(X,y):
    \begin{cases}
        size(X,p)\\
        element(y,X,p)
    \end{cases}
\end{equation}

% \newpage\subsection{Reverse}
\newpage\subsection{src.reverse()}

Consider the expression \inlineocl{src.reverse()}, where \inlineocl{src} is an expression resulting in a collection of objects or integers.
% Formal Definition:
% \begin{equation}\label{def:reverse}
%     reverse(X,Y,exp) \iff
% \end{equation}

Reformulation using Global Constraints:
\begin{equation}\label{csp:reverse}
    reverse(X,Y):
    \begin{cases}
        size(X,s)\\
        p_i = (s-i)*\llbracket i \leq s\rrbracket + i*\llbracket \neg(i \leq s)\rrbracket\\
        element(x_i,Y,p_i)
    \end{cases}
\end{equation}