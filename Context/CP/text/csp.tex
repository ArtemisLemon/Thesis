A CP model consists of a set of variables $X=\{x_1,\ldots, x_n\}$, a set of domains $\mathcal{D}$ mapping each variable $x_i\in X$ to a finite set of possible values $dom(x_i)$, and a set of constraints $\mathcal{C}$ on $X$, where each constraint $c(X(c),R)$ defines a set of values that a subset of variables $X(c)$ can take.

% A constraint $Ci = ( X_i , R_i )$ {\displaystyle C_{i}=({\mathcal {X}}_{i},{\mathcal {R}}_{i})} is defined by a set X i = { x i 1 , … , x i k } {\displaystyle {\mathcal {X}}_{i}=\{x_{i_{1}},\dots ,x_{i_{k}}\}} of variables and a relation R i ⊆ D i 1 × ⋯ × D i k {\displaystyle {\mathcal {R}}_{i}\subseteq {\mathcal {D}}_{i_{1}}\times \dots \times {\mathcal {D}}_{i_{k}}} that defines the set of values allowed simultaneously for the variables of X i {\displaystyle {\mathcal {X}}_{i}}.
%%
Domains can be either bounded, defined as an interval $\{lb..ub\}$, or enumerated, explicitly listing all possible values (e.g., ${1, 10, 100, 1000}$). This distinction impacts the choice of constraints: for instance, the global cardinality constraint is more effective with enumerated domains. 
%%
An assignment on a set $Y \subseteq X$ of variables is a mapping from variables in $Y$ to values in their domains. A solution is an assignment on $X$ satisfying all constraints.

Constraint programming also provides for other types of discrete domains for variables, such as Set and Graph variables.
Sets domains often use enumerations as lower and upper bound, to represent what \textit{must} be in the set, and what \textit{might} be in the set.
Graph domains generally employ set variables to represent links from a node in a graph.
For models of which the variables domains are boolean or real are covered in fields distinct from constraint programming, such as \textit{Boolean Satifiability}, \textit{Satifiability Modulo Theories} and \textit{Linear Programming}.