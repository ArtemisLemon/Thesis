% \subsection{Model Search}
Tools such as EMF have a need for tools assisting in the modeling process.
Model search is a powerful tool.

One of the open questions in MDE is \textit{how to effectively find models that conform to a metamodel}.
When the metamodel has model constraints the problem may become com.

\subsection{Model Search}
In \ytodo{cite Kleiner} there is a formal definition for model search.

\noindent
\textbf{Relaxed Metamodel}
A relaxed metamodel is a metamodel for which a subset of the constraints are not enforced.
These constraints include property cardinalities and model constraints.

\noindent
\textbf{Partial Model}
A partial model conforms to a relaxed metamodel, and partially conforms to the metamodel which was relaxed.
This generally means it is populated with Objects, but missing information on the values of referrences and attributes.
$$\exists o \in \texttt{Object} | o\texttt{.prop} = \emptyset$$
For a \inlineocl{Person}, this means not having their age, name, or know which other people they are related to.

\noindent
\textbf{Partially-conforms-to}
The relation between a Partial Model and the metamodel which was relaxed.

\noindent
\textbf{Model Search}
\begin{figure}[!ht]
    \centering
    % \begin{tikzpicture}
    %     \coordinate [label=180:{partial model}] (PM) at (0,0);
    %     \coordinate [label=0:{model}]  (M) at (10,0);
    %     \coordinate [label=180:{relaxed metamodel}]  (RMM) at (0,4);
    %     \coordinate [label=0:{metamodel}]  (MM) at (10,4);
      
    %       \fill (PM) circle (3pt);
    %       \fill (M) circle (3pt);
    %       \fill (MM) circle (3pt);
    %       \fill (RMM) circle (3pt);
          
    %     \draw[->, shorten <= 5pt, shorten >= 5pt,] (MM) -- (RMM) node [midway,fill=white] {relaxation};
    %     \draw[->, shorten <= 5pt, shorten >= 5pt,] (PM) -- (M) node [midway,fill=white] {model-search};
    %     \draw[->, shorten <= 5pt, shorten >= 5pt,] (PM) -- (RMM) node [midway,fill=white] {conforms-to};
    %     \draw[->, shorten <= 5pt, shorten >= 5pt,] (M) -- (MM) node [midway,fill=white] {conforms-to};
    %     \draw[->, shorten <= 5pt, shorten >= 5pt,] (PM) -- (MM) node [midway,fill=white] {partially-conforms-to};
    %     % \draw[->, shorten <= 5pt, shorten >= 5pt,] (M3) -- (M3) node [midway,fill=white] {conforms-to};
    %     % \draw (1pt,\y cm) -- (-1pt,\y cm) node[anchor=east] {$\mathbf{\y}$};
    %     % \path[->] (M3) edge [loop left] node {conforms-to} ();
    % \end{tikzpicture}
    \includegraphics[trim={0 0 0 0},width=1\linewidth]{Context/MDE/figures/modelsearch.png}
    \caption{Model Search}
    \label{fig:modelsearch}
\end{figure}


\subsection{Applications of Model Search}

\noindent
\textbf{Model Instantiation} is the process of generating a model that conforms to a metamodel.
Generally this means providing a partial model, with all the object instances.
The count of objects per class is often a parameter for model instanciation. 

\noindent
\textbf{Model Completion} is the process of taking a model with missing information and infering possible information. 
Generally this means providing a partial model, with all the object instances and some immutable data for the objects.
The missing data is the reason the model only partially conforms.

\noindent
\textbf{Model Repair} is the process of making a conforming model from a non-conforming model.
Repairing a model using model search implies creating a partial model by removing information from the object properties.
Automatically removing objects doesn't align with with our model repair use-cases, and is generally uncommon.

\noindent
\textbf{model space exploration}: classical model transformation tools associate one source model to one target model.
Model space exploration involves transformations transformations towards sets of models.
Domain Space Exploration describes a system of model transformations, model space exploration uses transformations over sets to model the system of model transformations.