\newpage
\subsection{Generic Example}
\begin{figure}[!ht]
    \centering
    \includegraphics[trim={0 0 0 0},width=1\linewidth]{Articles/ICTAI2025/figures/metamodel.pdf}
    \caption{UML Class Diagram as Metamodel}
\end{figure}
This figure present a generic class specification. It describes a class named \inlineocl{Object}, which has two properties: 
\inlineocl{attribute}: a collection of integers, with at least one and at most $m$ elements, \inlineocl{reference}: a collection of up to $n$ references to other \inlineocl{Object} instances.
%\inlineocl{reference} a collection of up to $n$ objects and \inlineocl{attribute} a collection of at least one and at most $m$ integers.
These illustrate the two main types of properties in object-oriented modeling:
Attributes, which store intrinsic data values (e.g., numbers or strings),
References, which define relationships between objects in the model.

This example presents a generic metamodel and model. We'll sometimes color it using the \inlineocl{Person} example:
$$\inlineumlclass{Person}{\inlineumlprop{age}{Int}{1}{1}, \inlineumlprop{children}{Person}{0}{*}}$$
$$\forall p,q \in \texttt{Person}, p \in q\texttt{.children} \implies p\texttt{.age} < q\texttt{.age}$$
