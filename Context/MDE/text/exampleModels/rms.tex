\newpage
\subsection{Reconfigurable Manufacturing Example}
\begin{figure}[t]
    \includegraphics[trim={0 0 0 0},scale=0.8]{Articles/SAC2025/figures/uc_uml.pdf}
    \caption{Class Diagram for RMS Task constraints} \label{fig:uc_uml}
    \Description[class diagram of RMS]{}
\end{figure}

This figure uses a class diagram to describe the concepts of an RMS, and how they relate (inspired by~\cite{Wang2012-srms}). In this figure, we focus on the graph structure of the model (classes and references among them), omitting attributes. We use the class diagram flavor from the Eclipse Modeling Framework (EMF)\footnote{\url{https://eclipse.dev/modeling/emf/}}, that connects classes by unidirectional \emph{references}, instead of bidirectional \emph{associations}. 

% \footnote{In the use case we're taking inspiration from, all the machines had the same characteristics. We chose an arbitrary number of characteristics to illustrate possible sizes of the resulting CSPs.}
%From it we take 
%the number of stages and tasks to get information about problem sizes, 
%the task precedence tree,
%and an estimate number of machines.
% Larger than the number of stages, smaller than the number of machines.
% equal to the max cardinality of our problem. 
% \ttodo{last two sentences read weird ?}
% \ttodo{I don't understand last part of this sentence}
% Because of the constant nature of the relations linking tasks, machines and characteristics, the optimal implementation doesn't ask the CP solver to compute those matches. 
% In light of this, for our particular use case,  the number of characteristics doesn't have to effect the final size of the CSP. 
% \ttodo{werll, this could be rewritten in a simpler way}

The two main components of a reconfigurable manufacturing system are stages, and machines which are organised into stages.
% Stages have three properties of which two are attributes (stageId and maxMachines) and one is a containment reference to machines.
% Stages are totally ordered and have a numeric label to materialise it, the stageID.
% Common practice in RMS is to label them 10, 20, and so on.
% Thus we represent it as an attribute.
% Additionally, each stage has a maximum number of machines.
A \texttt{Machine}'s property is its relation to a set of \texttt{Characteristics}.
% and for each machine a cost attribute.
% All the machines of a stage must share the same characteristics.
Objects of type \texttt{Task} are partially ordered, as expressed by the \texttt{prev} reference.
%, this order introduces precedence constraints, not addressed here.
% modeled by the prev reference.
% Which  to our problem.
Tasks have two other properties: a reference to a \texttt{Stage} (allocating the task to that stage), and a reference to characteristics (i.e., the machine characteristics needed to perform the task). Similarly to the example in \cite{Wang2012-srms}, tasks and machines can be linked to any number of characteristics.

\subsubsection {OCL Constraints for RMS}

\begin{listing}
\begin{lstlisting}[language=ocl]
context Task inv SameCharacteristicConstraint: 
    self.stage.machines.forall(m | m.characteristics
        ->includesAll(self.characteristics)) 

context Task inv PrecedenceConstraint: 
    self.stage.stageNum >= self.prev.stage.stageNum
\end{lstlisting}
\caption{RMS Task constraints in OCL.} \label{lst:uc_ocl}
\end{listing}