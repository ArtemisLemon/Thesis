% La page de garde est en français
% The front cover is in French
\selectlanguage{english}

% Inclure les infos de chaque établissement
% Include each institution data
\input{./Couverture-these/liste-ecoles-etablissements.tex}

% Inclure infos de l'école doctorale
% Include doctoral school data
\ecoledoctorale{SPIN}

% Inclure infos de l'établissement
% Include institution data
\etablissement{IMTA}

%Inscrivez ici votre sp\'{e}cialit\'{e} (voir liste des sp\'{e}cialit\'{e}s sur le site de votre \'{e}cole doctorale)
%Indicate the domain (see list of domains in your ecole doctorale)
\spec{Sciences et technologies de l'information et de la communication}

%Attention : le pr\'{e}nom doit être en minuscules (Jean) et le NOM en majuscules (BRITTEF) 
%Attention : the first name in small letters and the name in Capital letters 
\author{Matthew COYLE}

% Donner le titre complet de la th\`{e}se, \'{e}ventuellement le sous titre, si n\'{e}cessaire sur plusieurs lignes 
%Give the complete title of the thesis, if necessary on several lines
\title{Object Oriented Constraint Programming}
\lesoustitre{Models of UML and OCL semantics using Constraint Programming}

%Indiquer la date et le lieu de soutenance de la th\`{e}se 
%indicates the date and the place of the defense 
\date{« date »}
\lieu{IMT Atlantique Nantes}

%Indiquer le nom du (ou des) laboratoire (s) dans le(s)quel(s) le travail de th\`{e}se a \'{e}t\'{e} effectu\'{e}, indiquer aussi si souhait\'{e} le nom de la (les) facult\'{e}(s) (UFR, \'{e}cole(s), Institut(s), Centre(s)...), son (leurs) adresse(s)... 
%Indicates the name (or names) of research laboratories where the work has been done as well as (if desired) the names of faculties (UFR, Schools, institution...
\uniterecherche{« voir README et le site de de votre \'{e}cole doctorale »}

%Indiquer le Numero de th\`{e}se, si cela est opportun, ou laisser vide pour faire disparaitre cet ligne de la couverture
%Indicate the number of the thesis if there is one. otherwise leave empty so the line disappeurs on the cover
% \numthese{« si pertinent »} % \numthese{}

%Indiquer le Pr\'{e}nom en minuscules et le Nom en majuscules, le titre de la personne et l’\'{e}tablissement dans lequel il effectue sa recherche  
%Indicates the first name on small letters and the Names on capital letters, the person's title and the institution where he/she belongs to.
%Exemples :  Examples :
%%%- Professeur, Universit\'{e} d’Angers 
%%%- Chercheur, CNRS, \'{e}cole Centrale de Nantes 
%%%-  Professeur d’universit\'{e} – Praticien Hospitalier, Universit\'{e} Paris V  
%%%-  Maitre de conf\'{e}rences, Oniris 
%%%- Charg\'{e} de recherche, INSERM, HDR, Universit\'{e} de Tours  
 %S’il n’y a pas de co-direction, faire disparaitre cet item de la couverture  
 %In there is no co-director, remove the item from the cover
\jury{
{\normalTwelve \textbf{Rapporteur\textperiodcentered trice\textperiodcentered s avant soutenance :}}\\ \newline
\footnotesizeTwelve
\begin{tabular}{@{}ll}
Pr\'{e}nom NOM & Fonction et \'{e}tablissement d'exercice \\
Pr\'{e}nom NOM & Fonction et \'{e}tablissement d'exercice \\
Pr\'{e}nom NOM & Fonction et \'{e}tablissement d'exercice \\
\end{tabular}

\vspace{\baselineskip}
{\normalTwelve \textbf{Composition du Jury :}}\\
{\fontsize{9.5}{11}\selectfont {\textcolor{red}{\textit{Attention, en cas d’absence d’un\textperiodcentered e des membres du Jury le jour de la soutenance, la composition du jury doit être revue pour s’assurer qu’elle est conforme et devra être répercutée sur la couverture de thèse}}}}\\ \newline
\footnotesizeTwelve
\begin{tabular}{@{}lll}

Pr\'{e}sident\textperiodcentered e :        & Pr\'{e}nom NOM & Fonction et \'{e}tablissement d'exercice \textit{(à préciser après la soutenance)} \\
Examinateur\textperiodcentered trice\textperiodcentered s :         & Pr\'{e}nom NOM & Fonction et \'{e}tablissement d'exercice \\
                       & Pr\'{e}nom NOM & Fonction et \'{e}tablissement d'exercice \\
                       & Pr\'{e}nom NOM & Fonction et \'{e}tablissement d'exercice \\
                       & Pr\'{e}nom NOM & Fonction et \'{e}tablissement d'exercice \\
Dir. de th\`{e}se :    & Samir LOUDNI & Fonction et \'{e}tablissement d'exercice \\
Co-dir. de th\`{e}se : & Massimo TISI & Fonction et \'{e}tablissement d'exercice \textit{(si pertinent)} \\
Co-dir. de th\`{e}se : & Th\'{e}o LE CALVAR & Fonction et \'{e}tablissement d'exercice \textit{(si pertinent)} \\
\end{tabular}

\vspace{\baselineskip}
{\normalTwelve \textbf{Invit\'{e}\textperiodcentered e\textperiodcentered (s) :}}\\ \newline
\footnotesizeTwelve
\begin{tabular}{@{}ll}
Pr\'{e}nom NOM & Fonction et \'{e}tablissement d'exercice \\
\end{tabular}
}


\maketitle
